\documentclass[12pt,letterpaper]{article}
\usepackage{graphicx}
\usepackage{amssymb}
\usepackage{amstext}
\usepackage{amsmath}
\usepackage{setspace}

\begin{document}

\doublespacing

\section{Simple drifter dynamics}

In absence of ambient flow, a submerged drogue is forced by 
wave-induced velocities relative to the drogue motion.
For a monochromatic wave of mild slope and small amplitude $a$,
water surface has a shape $\eta = acos{(kx - \omega t)}$, 
and associated velocities:

\begin{equation}
u(x,z,t) = a \omega e^{kz} cos{(kx-\omega t)}
\end{equation}

\begin{equation}
w(x,z,t) = a \omega e^{kz} sin{(kx-\omega t)}
\end{equation}
$\omega$ is angular frequency; $k$ is wavenumber.
The above is true in deep water. For water of arbitrary depth $d$,
$e^{kz}$ is replaced with $cosh(k(z+d))/sinh(kd)$.

The acceleration of water particles induced by the pressure gradient
due to the surface slope is then:

\begin{equation}
a_x(x,z,t) = \dfrac{du}{dt} = a \omega^2 e^{kz} sin{(kx-\omega t)}
\end{equation}

\begin{equation}
a_z(x,z,t) = \dfrac{dw}{dt} = -a \omega^2 e^{kz} cos{(kx-\omega t)}
\end{equation}

The drag force induced by the flow on a submerged object such as drifter drogue
can be appoximated as:

\begin{equation}
F_d(x,z,t) = \dfrac{1}{2} \rho C_D \mathbf{u}_{rel}^2(x,z,t) A 
\end{equation}
where $\rho$ is water density; $C_D$ is drag coefficient of the drogue;
$\mathbf{u}_{rel}$ is the velocity of the flow relative to the drogue 
and it can be of either sign;
$A$ is the suface area perpendicular to $\mathbf{u}_{rel}$.
Here, $(x,z,t)$ refers to the position of drifter in time and space.
Because of wave phase evolution in time, the water acceleration due 
to pressure gradient from the surface slope is embedded in 
$\mathbf{u}_{rel}^2(x,z,t)$.

Since the wave-induced velocity field is strongly sheared near the surface
where the drifter is, a vertical integral of the velocity is considered to be 
acting on the drogue:

\begin{equation}
u = \dfrac{1}{L} \int_{z_D-0.5L}^{z_D+0.5L} u\ dz
= \dfrac{a \omega}{kL} \left( e^{k(z_D+0.5L)} - e^{k(z_D-0.5L)} \right)
cos{(kx-\omega t)}
\end{equation}
where $L$ is the drogue height and $z_D$ is the depth of of the drogue center.

If the drogue of surface area $A$ has some tilt $\phi$, which corresponds
to the deflection of its vertical axis away from $z$-axis, then the 
forces acting on the drogue in $x$ and $z$ are:

\begin{equation}
F_x = sgn(\mathbf{u}_{rel}) \dfrac{1}{2} \rho C_D \mathbf{u}_{rel}^2 A cos{\phi}
\label{eq_Fx}
\end{equation}

\begin{equation}
F_z = sgn(\mathbf{w}_{rel}) \dfrac{1}{2} \rho C_D \mathbf{w}_{rel}^2 A sin{\phi}
\label{eq_Fz}
\end{equation}
The tilt $\phi$ is positive when the drogue is tilted in the forward direction
of the wave, so $F_z$ vanishes if the drogue is perfectly vertical. 

A tilted drogue also projects some of the horizontal force into the vertical
and vice versa, so in fact the force experienced by the drogue is:

\begin{equation}
F_{dx} = F_x cos{\phi} -  F_z sin{\phi}
\label{eq_Fdx}
\end{equation}

\begin{equation}
F_{dz} = -F_x sin{\phi} + F_z cos{\phi}
\label{eq_Fdz}
\end{equation}

Combine equations (\ref{eq_Fx})-(\ref{eq_Fdz}) to obtain:

\begin{equation}
F_{dx} = \dfrac{1}{2} \rho C_D A 
\left[
 sgn(u_{rel}) \mathbf{u}_{rel}^2 cos^2{\phi} 
-sgn(w_{rel}) \mathbf{w}_{rel}^2 sin^2{\phi}
\right]
\end{equation}

\begin{equation}
F_{dz} = \dfrac{1}{2} \rho C_D A 
\left[
-sgn(u_{rel}) \mathbf{u}_{rel}^2 
+sgn(w_{rel}) \mathbf{w}_{rel}^2
\right]
sin{\phi} cos{\phi}
\end{equation}

Assumptions made:

\begin{enumerate}

\item Linear wave theory holds;

\item Force on an object can be approximated using drag coefficient;

\item Drifter drogue does not disturb the flow;

\end{enumerate}

\end{document}

